\chapter{Conclusion}
\label{chap:Conclusion}
In this thesis, an adaptive and interactive mobile application for smart
cafeteria has been proposed which will provide essential services to students at
everyday life in university. The main objectives were (I) Create ``Smart
Cafeteria'' which will be supported by web 2.0 systems and Smartphone
application. (II) The application should be adaptive and interactive. These
objectives has been achieved by finding stakeholders, functional requirements;
developing prototype for desktop and mobile platforms and usability evaluation
using HCI interaction design methodology for ``Smart Cafeteria''.


In depth, there were some sub goals of ``Smart Cafeteria'' which overlapped with
the main objectives and those were (i) Provide online cafeteria services, (ii)
Provide dieting services to the students, and (iii) Provide social collaboration
services into the system application. To achieve those objectives and sub goals,
I have analyzed requirements, developed prototypes for desktop and mobile
related to these services: (i) Mensa Queue Skipper, (ii) Menu Finder, (iii) Menu
Suggester and Dieting Adviser, (iv) Create Customized Menu and (v) Lunch with
Friend.


The online cafeteria services will decrease the length of queue in the cafeteria
and users' able to browse, choose and order meal in advance before entering
cafeteria. In addition, users will also be able to know time schedule of
cafeteria using this application. Dieting services provides some functionalities
such as generate dieting report for users, suggesting food menu based on users
choice and calorie consumption everyday in cafeteria, generate offers and
notification through email and sms which will help students stay happy and live
a healthy life. In the social collaboration, users share meal activities, status
with friends.


In this thesis, ``Smart Cafeteria'' has been analyzed, proposed the application,
designed both UI prototype for desktop and mobile, and finally usability
evaluation has been performed by user study. From the user evaluation, it is
proven that it is essential to provide quality services in cafeteria to achieve
the goal of a smart campus. By this proposed cafeteria application's ideas and
functionalities, the goals of ``Smart Cafeteria'' have been achieved.


\section{Further Work}
\label{sec:FurtherWork}
Since this is the first time that analysis and design has been performed in the
university cafeteria domain with the aim to provide services through web browser
or mobile devices, it is very important to design high fidelity prototype,
programming model such as ORM model, controller, view and service oriented
architecture (SOA) to provide real services to users.


At this moment, I have developed UML model and UI of application to demonstrate
``Smart Cafeteria'' and evaluated user experience following interactive design
methodology. In future I would like to extend the work and build high fidelity
prototype, develop mobile services and perform users study to evaluate the
system again until reaching maximum level of usability. I will also do more work
with application's adaptation using various machine learning approaches and
figure out the best approach for ``Smart Cafeteria''. Finally I will try to
develop rest of the parts as full functional and more interactive application of
``Smart Cafeteria'' which will ensure adaptability and usability.

