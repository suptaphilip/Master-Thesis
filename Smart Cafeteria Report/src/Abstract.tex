\begin{abstract}
\phantomsection
\addcontentsline{toc}{chapter}{Abstract}
\thispagestyle{plain}
\pagenumbering{roman}
\setcounter{page}{1}
% \noindent
Mobile Human-Computer Interaction is the relationship (interaction) between
people and their handheld mobile systems and the applications which they use in
their everyday life. In one word, mobile applications are interactive products
to support users in their day to day life no matter where they are. Since
technology is moving fast and changing rapidly, it is very important to
understand the mobile interaction techniques and also the possible impact of
mobile technologies in human life. Mobile HCI concerns about mobile applications
and mobile platforms which discusses the different possible interaction
techniques with those applications. The university cafeteria provides services
manually to the students which causes different kind of problems such as losing
time in cafeteria queue, unable to browse and choose meal in advance, not
providing any kind of dieting advices to students which helps them to follow
proper diet. In this thesis, I have proposed an adaptive and interactive mobile
application called ``Smart Cafeteria'' which will provide services to the
university students and faculty members to manage their meal in the university
cafeterias which in consequence makes their life more easy and comfortable. I
have followed HCI interaction design methodology in every step such as problem
define, data gathering, requirement analysis, prototype development and
usability evaluation for ``Smart Cafeteria'' application.\\ \\
\textbf{Keywords:} Mobile Human-Computer Interaction, Adaptive HCI, Interactive
Mobile Application, Smart Cafeteria.
\end{abstract}
\newpage