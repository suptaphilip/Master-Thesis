\begin{abstract}
\phantomsection
\addcontentsline{toc}{chapter}{Abstract}
\thispagestyle{plain}
\pagenumbering{roman}
\setcounter{page}{1}
% \noindent
Mobile Human-Computer Interaction is the relationship (interaction) between
people and their handheld mobile systems and the applications which we use in
our everyday life. In a word, mobile applications are interactive products to
support users in their day to day life no matter where they are. Since
technology is moving fast and changing rapidly, it is very important to
understand the mobile interaction techniques and the impact of possible mobile
technologies on human life. Mobile HCI concerns about the mobile applications
and PDA which discusses the different possible interaction techniques with those
applications. In the university cafeteria provides services manually to the
students which cause different kinds of problems such as losing time in
cafeteria queue, unable to browse and choose meal in advance, does not provides
any kind of dieting advices to students which help them to keep proper dieting.
In this thesis, I have proposed an adaptive, interactive mobile application
which is called ``Smart Cafeteria'' that will provide services to the university
students and faculties for managing their meal in the universities' cafeterias
which in consequence makes their life more easy and comfortable. In this work
from beginning to end, I have followed HCI interaction design methodology in
every step such as problem define, data gathering, requirement analysis,
prototype development and usability evaluation for ``Smart Cafeteria''
application.\\ \\
\textbf{Keywords:} Mobile Human-Computer Interaction, Adaptive HCI, Interactive
Mobile Application, Smart Cafeteria.
\end{abstract}
\newpage